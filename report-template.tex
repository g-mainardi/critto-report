\documentclass[a4paper,12pt]{report}

\usepackage{alltt, fancyvrb, url}
\usepackage{graphicx}
\usepackage[utf8]{inputenc}
\usepackage{float}
\usepackage{hyperref}

\usepackage{amssymb}
\usepackage{amsmath}

% Questo commentalo se vuoi scrivere in inglese.
\usepackage[italian]{babel}

\usepackage[italian]{cleveref}
\bibliographystyle{plain}
\bibliography{references}

\linespread{1.28} % Imposta l'interlinea di x volte la dimensione del carattere

\title{Teoria dei numeri \\ applicata alla Crittografia}

\author{Giosuè Giocondo Mainardi\\ (n.0000933566)}
\date{\today}


\begin{document}

\maketitle

\tableofcontents
\chapter{Introduzione}
La \textbf{crittografia} è fondamentale per garantire la sicurezza delle comunicazioni digitali, proteggendo la \emph{confidenzialità, l'integrità e l'autenticità} dei dati scambiati. 
Molti degli algoritmi crittografici più diffusi e robusti si basano su solidi principi matematici derivanti dalla \textbf{teoria dei numeri}. 

Sebbene in passato crittografia e matematica fossero viste come discipline separate, con \emph{G.H. Hardy} che nel 1940 vedeva la matematica come una scienza neutra, pura e gentile,
che non si schiera e non ha applicazioni dirette, l'avvento dell'informatica ha dimostrato il loro stretto legame. 

La teoria dei numeri, con lo studio delle proprietà dei numeri interi, dei \textbf{numeri primi} e delle loro interazioni, è diventata un pilastro portante per lo sviluppo di sistemi crittografici sicuri.

La \textbf{fattorizzazione in numeri primi}, il \textbf{logaritmo discreto} e le \textbf{equazioni diofantee} sono solo alcuni dei concetti chiave della teoria dei numeri che trovano 
applicazione nella crittografia moderna. Algoritmi rivoluzionari come \emph{RSA} e \emph{Diffie-Hellman}, basilari per la crittografia a chiave pubblica, sfruttano questi principi 
per garantire elevati livelli di sicurezza. La casualità è una caratteristica fondamentale in crittografia, eppure, come sottolineava \emph{Von Neumann}, 
``Chiunque consideri metodi aritmetici per produrre cifre casuali è, ovviamente, in stato di peccato'', evidenziando l'impossibilità di ottenere vera casualità in aritmetica. 
Tuttavia, la teoria dei numeri offre proprietà ideali per nascondere e proteggere informazioni, come la difficoltà nel risolvere alcuni problemi aritmetici nonostante la loro apparente semplicità.

In questa relazione, esploreremo nel dettaglio il legame inscindibile tra crittografia e teoria dei numeri, analizzando le principali applicazioni di quest'ultima nella protezione dei dati digitali. 
Verranno approfonditi concetti come la fattorizzazione, il logaritmo discreto e le equazioni diofantee, al fine di comprendere appieno il \textbf{ruolo cruciale} svolto dalla matematica nella \textbf{sicurezza informatica}.
%
%
%
%
\chapter{Fondamenti della teoria dei numeri}
È importante fornire prima basi teoriche matematiche per comprendere il funzionamento degli algoritmi che andremo a trattare.

La teoria dei numeri è lo studio dei numeri interi, e delle loro proprietà. L'insieme dei numeri interi è $\mathbb{Z}$ e contiene tutti i numeri interi positivi e negativi, insieme allo zero.
\begin{quote}
	\centering
	..., -5, -4, -3, -2, -1, 0, 1, 2, 3, 4, 5, ...
\end{quote}
\section{Definizioni e concetti fondamentali della teoria dei numeri.}

\subsection*{Numero primo}
Un numero primo è un numero intero \( p > 1 \) che ha esattamente due divisori: 1 e \(p\) (se stesso).

\subsection*{Numero composto}
Un numero composto è un numero intero \( n > 1 \) che ha almeno un altro divisore, oltre 1 e \( n \) (se stesso).

\subsection*{Fattorizzazione}
La fattorizzazione è il processo di scomposizione di un numero intero \( n \) in un prodotto di numeri primi. Formalmente, data un numero intero \( n \), la fattorizzazione di \( n \) è data dalla rappresentazione \( n = p_1^{e_1} \cdot p_2^{e_2} \cdot \ldots \cdot p_k^{e_k} \), dove \( p_1, p_2, \ldots, p_k \) sono numeri primi distinti e \( e_1, e_2, \ldots, e_k \) sono esponenti positivi.

\subsection*{Concetti di divisibilità e criteri di divisibilità}

Un numero \(a\) è divisibile per un numero \(b\) se il resto della divisione di \(a\) per \(b\) è zero. In altre parole, se esiste un numero intero \(k\) tale che \(a = b \cdot k\), allora \(a\) è divisibile per \(b\).

\begin{itemize}
	\item \textbf{Criterio della divisibilità per 2}: Un numero è divisibile per 2 se il suo ultimo cifra è pari, cioè se termina con 0, 2, 4, 6 o 8.
	
	\item \textbf{Criterio della divisibilità per 3}: Un numero è divisibile per 3 se la somma delle sue cifre è divisibile per 3.
	
	\item \textbf{Criterio della divisibilità per 5}: Un numero è divisibile per 5 se termina con 0 o 5.
	
	\item \textbf{Criterio della divisibilità per 7}: Un numero è divisibile per 7 se il numero ottenuto sottraendo il doppio della cifra delle unità dal numero ottenuto togliendo la cifra delle unità è divisibile per 7.

	\item \textbf{Criterio della divisibilità per 11}: Un numero è divisibile per 11 se la differenza tra la somma delle cifre in posizione pari e la somma delle cifre in posizione dispari è divisibile per 11.

\end{itemize}

\subsection*{Massimo comune divisore (MCD)}
Il massimo comune divisore di due numeri interi \( a \) e \( b \), denotato come \( \mathbb{MCD}(a, b) \), è il più grande numero intero che divide entrambi \( a \) e \( b \) senza lasciare resto.

\subsection*{Minimo comune multiplo (mcm)}
Il minimo comune multiplo di due numeri interi \( a \) e \( b \), denotato come \( \mathrm{mcm}(a, b) \), è il più piccolo multiplo comune di \( a \) e \( b \).

\chapter{Aritmetica Modulare}

L'aritmetica modulare è un ramo fondamentale della teoria dei numeri che studia le operazioni aritmetiche sui resti delle divisioni intere. Questo capitolo esplora i concetti chiave dell'aritmetica modulare necessari per comprendere molti algoritmi crittografici moderni.

\section{Congruenza Modulare}

Data una coppia di interi $a$ e $b$, e un intero positivo $n$ detto \textbf{modulo}, diciamo che $a$ è congruo a $b$ modulo $n$, denotato come:

$$a \equiv b \pmod{n}$$

Se esistono interi $k_1$ e $k_2$ tali che:

$$a = k_1n + b$$
$$b = k_2n + a$$

In altre parole, $a$ e $b$ hanno lo stesso resto quando divisi per $n$. L'insieme di tutti gli interi congrui a $a$ modulo $n$ è chiamato \textbf{classe di resto} o \textbf{classe di congruenza} di $a$ modulo $n$, denotata come $[a]_n$.

La relazione di congruenza gode delle seguenti proprietà:

\begin{enumerate}
   \item \textbf{Riflessività}: $a \equiv a \pmod{n}$ per ogni $a \in \mathbb{Z}$ e $n > 0$.
   \item \textbf{Simmetria}: Se $a \equiv b \pmod{n}$, allora $b \equiv a \pmod{n}$.
   \item \textbf{Transitività}: Se $a \equiv b \pmod{n}$ e $b \equiv c \pmod{n}$, allora $a \equiv c \pmod{n}$.
\end{enumerate}

\section{Operazioni Aritmetiche Modulari}

Le operazioni aritmetiche fondamentali (addizione, sottrazione, moltiplicazione ed esponenziazione) possono essere definite sull'insieme dei residui modulo $n$, denotato come $\mathbb{Z}_n$. Per $a, b \in \mathbb{Z}_n$:

\begin{enumerate}
   \item \textbf{Addizione modulare}:
   $$(a + b) \bmod n = r \quad \text{dove } r \in \mathbb{Z}_n \text{ è il resto della divisione di } a+b \text{ per } n$$
   
   \item \textbf{Sottrazione modulare}:
   $$(a - b) \bmod n = r \quad \text{dove } r \in \mathbb{Z}_n \text{ è il resto della divisione di } a-b \text{ per } n$$
   
   \item \textbf{Moltiplicazione modulare}:
   $$(a \cdot b) \bmod n = r \quad \text{dove } r \in \mathbb{Z}_n \text{ è il resto della divisione di } a \cdot b \text{ per } n$$
   
   \item \textbf{Esponenziazione modulare}:
   $$a^b \bmod n = r \quad \text{dove } r \in \mathbb{Z}_n \text{ è il resto della divisione di } a^b \text{ per } n$$
\end{enumerate}

Queste operazioni godono delle stesse proprietà dell'aritmetica ordinaria, come la commutativà per l'addizione e la moltiplicazione, l'associatività, la presenza di elementi neutri (0 per l'addizione, 1 per la moltiplicazione) e di inversi (modulo $n$).

\section{Inverso Moltiplicativo Modulare}

Un concetto cruciale è quello di \textbf{elemento invertibile modulo $n$}: un numero $a \in \mathbb{Z}_n$ è invertibile se esiste $b \in \mathbb{Z}_n$ tale che $ab \equiv 1 \pmod{n}$. $b$ è detto l'\textbf{inverso moltiplicativo} di $a$ modulo $n$, denotato come $a^{-1}$.

L'inverso moltiplicativo di $a$ modulo $n$ esiste se e solo se $a$ e $n$ sono coprimi, ovvero il loro massimo comune divisore $\mathbb{MCD}(a,n) = 1$. 

L'algoritmo euclideo esteso permette di calcolare efficacemente $a^{-1}$ modulo $n$.

\section{Aritmetica Modulare sui Numeri Primi}

Le proprietà dell'aritmetica modulare si semplificano notevolmente quando il modulo $n$ è un numero primo $p$. In questo caso, l'insieme $\mathbb{Z}_p$ è un \textbf{campo}, ovvero ogni elemento non nullo è invertibile.

\subsubsection*{Piccolo Teorema di Fermat}

$$a^p \equiv a \pmod{p} \quad \forall a \in \mathbb{Z}_p$$

Questo significa che elevare un numero $a$ alla potenza $p$ modulo $p$ dà come risultato $a$ stesso.

\subsubsection*{Teorema di Eulero}

$$a^{\phi(n)} \equiv 1 \pmod{n} \quad \text{se } \mathbb{MCD}(a,n) = 1$$

Dove $\phi(n)$ è la \textbf{funzione totiente di Eulero}, che conta il numero di interi positivi minori di $n$ e coprimi con $n$.

Queste proprietà hanno implicazioni cruciali per la crittografia modulare su campi finiti, in particolare per la crittografia a chiave pubblica come RSA.

Nei prossimi capitoli, esploreremo ulteriormente i concetti di aritmetica modulare, fattorizzazione in numeri primi, funzione di Eulero e le loro applicazioni agli algoritmi crittografici.


\subsection*{Aritmetica Modulare}
L'aritmetica modulare è un ramo della teoria dei numeri che si occupa delle operazioni aritmetiche su numeri interi all'interno
di un insieme di resti modulo un numero fissato, noto come modulo. 

Inoltre, definisce un insieme di numeri interi con operazioni aritmetiche specifiche, che possono essere descritte come segue:

\subsubsection*{Definizione dell'insieme modulo}
Si sceglie un numero intero positivo \(m\), chiamato modulo, che determina l'insieme dei resti modulo \(m\). 
Questo insieme è denotato come \(\mathbb{Z}_m\) o \(\mathbb{Z}/m\), ed è composto da tutti i numeri interi che vanno da 0 a \(m-1\).

\subsubsection*{Congruenze}
Due numeri \( a \) e \( b \) sono congruenti modulo \( m \), denotato come \( a \equiv b \pmod{m} \), se \( a \) e \( b \) hanno lo stesso resto quando divisi per \( m \). Formalmente, \( a \equiv b \pmod{m} \) se e solo se \( m \) divide \( a - b \).

\subsubsection*{Definizione delle operazioni aritmetiche}
Sulle congruenze modulo \(m\), vengono definite le operazioni aritmetiche che rispettano le proprietà dell'insieme dei resti modulo \(m\):
\begin{itemize}
	\item \textbf{Addizione modulare}: \( (a + b) \mod m \)
	\item \textbf{Sottrazione modulare}: \( (a - b) \mod m \)
	\item \textbf{Moltiplicazione modulare}: \( (a \cdot b) \mod m \)
\end{itemize}

\subsubsection*{Proprietà dell'aritmetica modulare}
L'aritmetica modulare gode di alcune proprietà interessanti, tra cui la chiusura rispetto alle operazioni aritmetiche (cioè il risultato di un'operazione modulare è sempre un numero nell'insieme dei resti modulo \(m\)), l'associatività, la commutatività e la distributività.
%
%
%
\subsection*{Teorema cinese del resto}
Il teorema cinese del resto afferma che se \( m_1, m_2, \ldots, m_n \) sono interi positivi coprimi a coppie, e \( a_1, a_2, \ldots, a_n \) sono interi arbitrari, allora esiste un unico intero \( x \) che soddisfa il sistema di congruenze \( x \equiv a_1 \pmod{m_1}, x \equiv a_2 \pmod{m_2}, \ldots, x \equiv a_n \pmod{m_n} \).

\subsection*{Logaritmo discreto}
Il logaritmo discreto è definito come segue: date una base \( g \) e un modulo \( p \), il logaritmo discreto di \( b \) rispetto a \( g \) modulo \( p \), denotato come \( \log_g(b) \), è l'intero \( x \) tale che \( g^x \equiv b \pmod{p} \).

\subsection*{Curve ellittiche}
Le curve ellittiche sono insiemi di punti \( (x, y) \) che soddisfano un'equazione di forma cubica: \( y^2 = x^3 + ax + b \), dove \( a \) e \( b \) sono costanti. Le curve ellittiche hanno molte applicazioni in crittografia grazie alla loro proprietà di essere non solo additive ma anche moltiplicative.

\section{Sezione 1 Capitolo 2}

Contenuto.

\subsection*{Esempio}
Conseguentemente, GLaDOS è un ``observable'' per Output.

Qua esempio di immagine inserita sul posto.

\begin{figure}[h]
\centering{}
\includegraphics[width=\textwidth]{img/example_img.pdf}
\caption{L'interfaccia \texttt{GLaDOS}.}
\label{img:example}
\end{figure}

\section{Altra sezione}

\textbf{Sottotesto in grassetto}.

\subsection*{Esempio minimale}

\subsubsection{Sotto sotto sezione con paragrafi}

\paragraph{Paragrafo1} Contenuto.

\paragraph{Paragrafo Risposta 1} Contenuto \textit{paragrafo}, con riferimento immagine
\Cref{img:example}: e anche \texttt{Scrittura unicode}.

\chapter{Capitolo 3}
\section{Sezione}

Contenuto

\subsection{Esempio con sottosezioni e permalink}

\subsubsection{Utilizzo di \texttt{LoadingCache} dalla libreria Google Guava}

Permalink: \url{https://github.com/AlchemistSimulator/Alchemist/blob/d8a1799027d7d685569e15316a32e6394632ce71/alchemist-incarnation-protelis/src/main/java/it/unibo/alchemist/protelis/AlchemistExecutionContext.java#L141-L143}

\subsubsection{Utilizzo di \texttt{Stream} e lambda expressions}

Usate pervasivamente. Il seguente è un singolo esempio.
Permalink: \url{https://github.com/AlchemistSimulator/Alchemist/blob/d8a1799027d7d685569e15316a32e6394632ce71/alchemist-incarnation-protelis/src/main/java/it/unibo/alchemist/model/ProtelisIncarnation.java#L98-L120}

\subsubsection{Scrittura di metodo generico con parametri contravarianti}

Permalink: \url{https://github.com/AlchemistSimulator/Alchemist/blob/d8a1799027d7d685569e15316a32e6394632ce71/alchemist-incarnation-protelis/src/main/java/it/unibo/alchemist/protelis/AlchemistExecutionContext.java#L141-L143}

\chapter{Capitolo finale}

In quest'ultimo capitolo si tirano le somme del lavoro svolto e si delineano eventuali sviluppi
futuri.

\section{Sezione di commenti finali}

\textbf{È richiesta una sezione per ciascun membro del gruppo, obbligatoriamente}.

\appendix
\chapter{Capitolo appendice 1}

Contenuto.

\chapter{Capitolo appendice 2}

Contenuto.

\bibliographystyle{alpha}
\bibliography{report-template}

\end{document}
